\documentclass{article} \usepackage[margin=1in]{geometry}
\usepackage{pdfpages}
\begin{document}

\includepdf{response_letter.pdf}

\newpage

\section{Point-by-point responses}


\subsection{Editor}

\textbf{Comment: This is a very valuable and interesting manuscript.  It is very well
  suited to Field Crops Research.  All reviewers recommend 'Minor
  revision'.  There was some uncertainty over the statistical analysis
  and so a third reviewer was asked to review this aspect of it. This
  was not able to be entered into EVISE and so I have appended it below.}\\

\textbf{Response:} Thank you for including these comments. \\

\textbf{Comment: This paper analyses 33 years worth of single-site yield data on rice
  varieties grown in California. Regression models are fitted to assess
  yield trend due to cultivars and trend due to "yield erosion".I have a
  few comments on the statistical analysis.}\\

\textbf{(1) The term yield erosion suggests that this declining trend
  component relates to properties of the cultivars, or cultivars losing
  their potential over time. I am not sure this interpretation is the
  only one possible. There are many other factors that may cause a
  decline over time, i.e. climate change, deterioration of the soil,
  agronomic practices not keeping up with changes of the environment
  etc. See my comments further below.}\\

\textbf{Response:} We were using the term ``yield erosion'' as this is
the term used to describe this phenomonom in previous work in
rice. Given that ``erosion'' is a slow, sometimes
imperceptible, process of reducing something
driven by abiotic, climatic, and biotic forces, ``erosion'' seems to
be the appropriate term here. While there is a component of ``erosion'' inherent
to the material being eroded, this term is not often interpreted as it
has been here, and there are likely other concerns which are loading
this term with additional meaning.

However, given the sensitivities involved, we have elected to 
remove the term ``yield erosion'' throughout the manuscript.\\

\textbf{Comment: (2) "CRB design" is not a proper name for a design. I suspect the
  authors are refering to a randomized complete block design (RCBD).}\\

\textbf{Response:} Thank you for pointing out this typo - the correct
design is a CRD. The text has been update accordingly.\\

\textbf{Comment: (3) I think the model statement in equation (1) needs some polishing
  up. Indexing observations just by plots is not sufficient. It appears
  from the model description that plot data were modelled. If that's the
  case, then the model misses a block effect.}\\

\textbf{Response:} As stated above, there are no blocks, hence the
model is correct in this aspect.\\

\textbf{Comment: Moreover, a plot error term is lacking.}\\

\textbf{Response:} Added, though it should be noted that when detailing
the structural aspect of the model in an article, the error term is often
not included. \\

\textbf{Comment: For clarity, the authors could add subscripts for
  years, cultivars and blocks nested within years, amounting to three
  different subscripts.}\\

\textbf{Response:} Subscript for years and cultivar added. There are
no blocks, so this subscript is not needed.\\

\textbf{Comment: If plot data are analysed, a random year-cultivar interaction
  needs to be added as well.}\\

\textbf{Response:} Added to the model. The estimates are largely
unchanged from before.\\

\textbf{Comment: I do not think
  it's appropriate to equate the intercept with an overall mean, unless
  both time covariates were mean-centered, which does not seem to be the
  case.}\\

\textbf{Response:} The time variates were mean-centered as stated in the
manuscript. We have clarified this further (lines 91).\\

\textbf{Comment: (4) If subscripts are used for years and cultivars, the covariates
  should be indexed accordingly. If i and k index cultivars and years,
  respectively, then the time elapsed since release needs to be indexed
  by both i and k (a\_ik, say), whereas the year of release would be
  indexed by i only (r\_i, say). Using this notation, the authors are
  regressing yield on a\_ik and r\_i.}\\

\textbf{Response:} Fixed\\

\textbf{Comment: There's a simple third time
  covariate that could be considered for regression: calendar year
  (t\_k). This regression would directly assess non-genetic components
  of trend, due, e.g., to climate change, advances in agronomic
  practices etc. It turns out, however, that a regression on all three
  covariates is not possible, because a\_ik = t\_k - r\_i, meaning that
  there would be confounding. Thus, only two of these three regressor
  vaiables can be fitted, so the regression done by the authors is fine
  in pinciple.  Nevertheless, this little thought exercise shows that
  there is some ambiguity in the interpretation of regression
  coefficients. This issue is discussed in some detail in
  Piepho, H.P., Laidig, F., Drobek, T., Meyer, U. (2014): Dissecting
  genetic and non-genetic sources of long-term yield trend in German
  official variety trials. Theoretical and Applied Genetics 127,
  1009-1018. The authors of that paper did a regression on t\_k and r\_i.}\\

\textbf{Response:} Yes, this is indeed the case. However, this is
largely a methodological concern, and after some consideration, we have
elected to not discuss this matter here, due to space constraints.\\

\subsection{Reviewer 1}

\textbf{Comment: This paper certainly should be published as the results are
  important even if they lack a satisfactory explanation. But I would
  like to see some changes and more attention to some issues. Much is
  annotated on the attached manuscript, and some repetitive and
  confusing, those remarks came as I read the paper. This summary is
  upon more considered reflection afterwards.}

\textbf{There seems little doubt that for many cultivars their yield
  steadily declined with time, or years after release which was used
  here, but it is the same as simply years. Figure S1 says it all
  (that’s your whole data set I presume): 8 out of the 14 cvs showed
  yield decreases, the other 6 little change with years after release;
  this Figure should be in the paper proper up front I think, it is
  more convincing that the stats modelling derived Figs 3 and 4.}\\

\textbf{Response:} On the suggestion of several reviewers, we have
moved figure S1 into the main paper.\\

\textbf{Comment: It was important to test whether the interaction cv x years after
  release was significant, because this could give a clue as to what is
  the cause of the yield decline, but it wasn’t significant according to
  the stats which I didn’t really understand. Discussion could explore
  this in more depth; I mean why some cvs didn’t lose yield and some
  did. Maybe only widely adopted cvs lost yeild, due to greater presume
  for disease to evolve virulence?}\\

\textbf{Response:} The reviewer should mindful that
stating that the estimated CI for the cultivar-year interaction
includes zero is not the same thing as ``no
effect''. Rather, the correct interpretation is that the effect, if it
exists, is smaller than our power to detect. Therefore, the correct
conclusion is that the differences between cultivars in the decline
over time, if existent, is likely small, though the sign and magnitude of the exact
effect is uncertain.

While this is indeed interesting, we
have attempted to limit our speculation on this point given that we do not have
sufficient data to actually test many of these ideas. We have expanded
the discussion to include some of the points raised (lines 210). \\

\textbf{Comment: The paper seems to be written to put the blame on the cultivars and
  the challenge on the breeders, the notion of their cvs’ yield
  advantages being “eroded”. This might be the case if a biotic stress
  is evolving virulence against the host plant resistance of any cv.
  But it is not really the case when there is a deterioration in the
  aspects of the environment (climate, agronomy, and/or soil), against
  which younger cultivars are more resistant or tolerant.  For me the 29
  kg/ha/yr could be breeding progress in the face of the deteriorating
  E; at about 0.3\% p.a. it is not unlike the rate of progress in
  advanced wheat breeding programs (anything near the 1.2\% p.a. which
  the paper mentions in discussion is very rare these days).}\\

\textbf{Response:} The term ``yield erosion'' has been removed due to
these concerns. However, we have taken great pains to specifically
point out that the evidence suggests this inherent to the system
(as the title of this manuscript clearly states), that it is similar
to observations from other rice systems (lines 211-217), and that it is plausibly
due to the environmental aspects the reviewer lists (lines 190-210).

However, the evidence from this study explicitly states that given
these data and this analysis, a rate of improvement of 29 kg/ha/yr
is unlikely. The reviewer appears to have missed one of the main
conclusions of the manuscript. We have edited for greater clarity. \\

\textbf{Comment: Really it all depends on the definition of potential yield. See van
  Ittersum and colleagues in FCR Vol 143 March 2013. Its determined by
  latest best yielding Cv and the current E in the absences of any
  manageable stresses, both biotic and abiotic, and measured in good
  experiments and models based on such experiments. A breakdown in cv
  resistance is probably best described as “erosion”, while a
  deterioration in the E due, for example, to a negative climate trend,
  is a decrease in the potential yield, but a new cv tolerating this
  deterioration would lift the potential yield (see also below).}\\

\textbf{Response:} Added discussion of these issues (lines 174-181). \\

\textbf{Comment: The authors should read McKay et al (2010, TAG 122, 225-238) where a
  similar situation in winter wheat in the UK is described. But by
  having plots with and without fungicide over the last 40 years or so,
  these authors could measure the erosion and narrow down the cause
  (foliar disease), which did vary with cultivar.}\\

\textbf{Response:} This is an interesting paper (assuming the reviewer
intended \textbf{Mackay et al. 2011}), but unfortunately
these data do not lend themselves to similar explorations. Unlike
Mackay et al., we do not have paired treated/untreated plots. In the
course of writing this paper, we discussed various means to to narrow down the
cause of yield losses, but ultimately lack the data to do so. We have
added a citation to McKay et al. to the manuscript.\\

\textbf{Comment: Also the authors need to refer to McKenzie et al (2014) in the CSSA
  Special Publicn 33 with a whole section on rice yield progress in
  Ca.}\\

\textbf{Response:} Discussion added (lines 247-269). \\

\textbf{Comment: These authors show in Fig 10-8 trial results from all over the
  rice areas of Ca, whereas the manuscript deals with trials solely
  from the RES at Briggs.This needs to be discussed and may help
  explain what is happening, as there is little or no evidence in
  McKenzie et al that, for example, the yield of M-202 is declining
  with time.}\\

\textbf{Response:} Explaination added (lines 247-269)\\

\textbf{Comment: What also worries me is that Ca farm yields (and Butte
  county yields) have been increasing since 1995 or so, at a low rate
  (about 70 kg/ha/yr) but increasing, which doesn’t agree with Fig S1
  no matter what cv you choose.}\\

\textbf{Response:} The reviewer appears to misunderstand basic
differences in the data being presented. We agree and plainly state in
the manuscript that CA farm yields have been increasing (line
). However, we do not find evidence that gains as large as 70 kg/ha/yr
are due to increasing genetic yield potential. It is entirely possible
for farm yields to increase for other reasons which, again, we outline
specifically in the manuscript (lines 241-246).\\

\textbf{Comment: If yield is declining because of increasing disease susceptibility
  with time that is erosion, well known to wheat breeders, and driving
  what is called maintenance breeding. By definition yield potential is
  measured without biotic stresses, which are manageable, so this sort
  of erosion is not a loss of yield potential. If the yield
  deterioration is due to something like increased ozone pollution or
  cooler nights at meiosis, that is less clear cut: those stresses are
  part of the environment, largely unmanageable, and amount to a
  decrease in the yield potential of the environment, but breeding can
  still overcome them, at 29 kg/ha/yr if this was the cause in this rice
  study. Therefore the possible causes at BES need more discussion and
  reference to other trial data sources. Sorry this repeats 4 a bit!}\\

\textbf{Response:} As this repeats the previous concerns, please see
above for our response.\\

\textbf{Comment:  I am not sure, however, that you need to show Ca yields (Fig 1)
  because this broadens the discussion a lot, and almost demands you
  look at what is happening across the rice region in terms of weather
  trends etc, probably using a rice simulation model (although
  modelling at just RES itself would be relevant to the paper also). It
  is useful to show how close farm yields are to RES trial yields but
  that could be covered in a text sentence to two. And if you want to
  discuss this bigger picture, please include a simple map of rice
  areas in Ca.}\\

\textbf{Response:} Fig. 1 has been moved to supplemental material.\\

\textbf{Comment: Table 2 lists the key cv data and should be up front in the paper. You
  could add to the Table brief comments on the important quality attr
  ibutes of the various releases (if there any) and finally you could
  include a column showing the important data of Fig 2, which has
  nothing to do with the modelling.}\\

\textbf{Response:} Table 1 and 2 have been swapped. Quality attributes
are not of direct relevance to this paper, so we have provided a
citation for where this information can be found for those
interested (line 235). We disagree that the data from Figure 2 should be
presented in table form. Data of this nature is best presented in
figures, and presenting these data twice is redundent.\\

\textbf{Comment: So in summary, I am not sure you need Fig 1, you could put the data of
  Fig 2 in Table 2 and bring it to the front along with Fig S1, and I
  don’t think you need Fig 6. But consider how you use the word
  “erosion” and discuss more thoroughly what could be causing the yield
  declines in Fig S1, and whether there is evidence it is happening also
  outside of the RES at Biggs.}\\

\textbf{Response:} Figure 1 has been moved to supplemental material,
and figure S1 has been moved to the main body. See above regarding
putting the data from figure 2 in tabular format.\\

\subsection{Reviewer 2}

\textbf{Comment: Using data from variety trials, the authors proved the importance of
  maintenance breeding for world food security. However, I am not very
  familiar with the statistical analysis that the authors used to derive
  the conclusions. If a reviewer who is familiar in the related
  statistical procedures has no problem with the methodology, I believe
  it is suitable for publication in FCR. Please consider the following
  suggestion during your revision.}\\


\textbf{Comment: L10: Something is wrong with “…with it undesirable…”}\\

\textbf{Response:} Edited\\

\textbf{Comment: L30: Yield increase of 50-62 kg per ha per year was not slow.}\\

\textbf{Response:} Added clarification.\\

\textbf{Comment: L51-52: What are the two numbers (39.4648, - 121.7342) ? You may use
normal way to indicate the location.}\\

\textbf{Response:} These are standard latitude/longitude coordinates.\\

\textbf{Comment: L53: “Trials” instead of “Trails”}\\

\textbf{Response:} Corrected.\\

\textbf{Comment: L120: (Fig. 2)}\\

\textbf{Response:} Corrected.

\textbf{Comment: L141: Something is wrong with “…since release is between…”}

\textbf{Response:} Corrected.\\

\textbf{Comment: Fig. S1: It appears that there are two kinds of dots in the figure
(black and grey). Please indicate their difference.}\\

\textbf{Response:} All points are black.\\

\newpage

\includepdf{diffs.pdf}

\newpage

\includepdf{Espe_rice_yield_improvement_revised.pdf}

\end{document}
